\documentclass[../main]{subfiles}

\begin{document}
\begin{sloppy}

\section{搜索}

\subsection{BFS和DFS基础}

    构建二叉树
    \begin{verbatim}
               E
             /   \
            B     G
           / \   / \
          A   D F   I
             /     /
            C     H
    \end{verbatim}

\begin{lstlisting}[style = Python]
def build():
    a = node('a')
    b = node('b')
    c = node('c')
    d = node('d')
    e = node('e')
    f = node('f')
    g = node('g')
    h = node('h')
    i = node('i')
    root = e
    root.l = b
    root.r = g
    b.l = a
    b.r = d
    d.l = c
    g.l = f
    g.r = i
    i.l = h
    return root
    
\end{lstlisting}
    
    \subsubsection{BFS}

        代码实现BFS可以使用\textbf{队列}. 处理第i层的节点a时, 将a的第i+1层的所有孩子节点放入队尾.

\begin{lstlisting}[style = Python]
class node():
    def __init__(self, v=None, l=None, r=None):
        self.value = v
        self.l = l
        self.r = r

# using queue to BFS
root = build()
q = deque()
q.append(root)
while len(q) > 0:
    temp = q.popleft()
    print(temp.value, end=' ')
    if temp.l:
        q.append(temp.l)
    if temp.r:
        q.append(temp.r)

\end{lstlisting}
    
\begin{verbatim}
e b g a d f i c h
\end{verbatim}

        BFS遍历二叉树时, 每条边需要且只需要检查一次, 时间复杂度为O(m), m为边的数量; 每个点只进出队列一次, 空间复杂度为O(n), n为点的数量.

        BFS适用于寻找全局最优解. BFS编码时需要注意去重, 待搜索的状态已经搜素过时, 不再将此状态放入队尾.
        

        \subsubsection{DFS}

            代码实现DFS可以使用递归, DFS模板框架为

            \begin{lstlisting}[style = Pseudocode]
GLOBAL ans;
Function dfs(Layer NO., other parameters):
    IF (exit loop):
        ans <- updated ans;
        return; // back to previous layer
    ENDIF
    剪枝
    FOR (下一层所有可能情况):
        IF (used[i] == false): // have not been used
            used[i] = true // mark status
            dfs(layer + 1, other parameters)
            used[i] = false // recover status
        ENDIF
    ENDFOR
    return; // back to previous layer
            \end{lstlisting}

            \textbf{DFS常用操作}

            \begin{lstlisting}[style = Python]
            # DFS
root = build()

# 时间戳
dfn = []
dfn_timer = 0
def dfn_node(root):
    global dfn, dfn_timer
    if root:
        dfn_timer += 1
        dfn.append((root.value, dfn_timer))
        print('{}: {}'.format(root.value, dfn_timer), end=' ')
        if root.l:
            dfn_node(root.l)
        if root.r:
            dfn_node(root.r)

print('时间戳')
dfn_node(root)
print()

# DFS序
visit_timer = 0
def visit_order(root):
    global visit_timer
    if root:
        visit_timer += 1
        # 第一次 访问
        print('{}: {}'.format(root.value, visit_timer), end=' ')
        if root.l:
            visit_order(root.l)
        if root.r:
            visit_order(root.r)
        # 第二次 回溯
        visit_timer += 1
        print('{}: {}'.format(root.value, visit_timer), end=' ')

print('DFS序')
visit_order(root)
print()

# 树的深度
deep = dict()
deep_timer = 0
def deep_node(root):
    global deep_timer, deep
    if root:
        deep_timer += 1
        deep[root.value] = deep_timer
        print('{}: {}'.format(root.value, deep_timer), end=' ')
        if root.l:
            deep_node(root.l)
        if root.r:
            deep_node(root.r)
        deep_timer -= 1

print('树的深度')
deep_node(root)
print()

# 子树节点总数
num = dict()
def num_node(root):
    global num
    if not root:
        return 0
    else:
        num[root.value] = num_node(root.l) + num_node(root.r) + 1
        print('{}: {}'.format(root.value, num[root.value]), end=' ')
        return num[root.value]

print('子树节点总数')
num_node(root)
print()
            \end{lstlisting}

\begin{verbatim}
时间戳
e: 1 b: 2 a: 3 d: 4 c: 5 g: 6 f: 7 i: 8 h: 9 
DFS序
e: 1 b: 2 a: 3 a: 4 d: 5 c: 6 c: 7 d: 8 b: 9 g: 10 \
f: 11 f: 12 i: 13 h: 14 h: 15 i: 16 g: 17 e: 18 
树的深度
e: 1 b: 2 a: 3 d: 3 c: 4 g: 2 f: 3 i: 3 h: 4 
子树节点总数
a: 1 c: 1 d: 2 b: 4 f: 1 h: 1 i: 2 g: 4 e: 9 
\end{verbatim}
            
            
        
            DFS适用于寻找一个可行解. 编码时需要注意剪枝, 当某一个分支已经不符合要求时, 则不再在此分支上深入搜索. 


\newpage
\subsection{剪枝与判重}
    1. 可行性剪枝: 检查当前状态, 出现条件不合法则剪枝

    2. 搜索顺序剪枝: 搜索树有多个层次和分支, 不同的搜索顺序会造成不同的复杂度

    3. 最优性剪枝: 如果当前花费的代价已经超过前面搜索到的最优解, 退出本分支. EX. 当前路径已经长于目前搜索到的最短路径

    4. 排除等效冗余: 沿着当前节点搜索全部分支, 结果一样, 退出. 一般与组合问题有关, EX. 多个数字凑成一个大数, 数字的顺序不影响结果

    5. 记忆化搜索: 将已经计算出的结果保存起来, 常用于DP.

    \subsubsection{BFS判重\ 蓝桥642}

有9只盘子,排成1个圆圈。其中8只盘子内装着8只蚱蜢,有一个是空盘。我们把这些蚱蜢顺时针编号为1~8。

每只蚱蜢都可以跳到相邻的空盘中, 也可以再用点力,越过一个相邻的蚱蜢跳到空盘中。

请你计算一下,如果要使得蚱蜢们的队形改为按照逆时针排列, 并且保持空盘的位置不变(也就是1−8换位,2−7换位,...),至少要经过多少次跳跃?

\textbf{化圆为线}, 将空盘子标记为0, 从12点方向依顺时针将蚂蚱编号合并为字符串, 初始状态为012345678. 若1跳到盘子里, 变为1023456789; 若7越过8跳到盘子里, 变为712345608. 目标状态为087654321.

\begin{lstlisting}[style = Python]
from collections import deque

q = deque()
mapp = set()

def solve():
  global q, mapp
  while len(q) > 0:
    now = q[0]
    q.popleft()
    s = now[0]
    step = now[1]
    if s == '087654321':
      print(step)
      return
    idx = 0
    for i in range(9):
      if s[i] == '0':
        idx = i
        break
    for j in range(idx-2, idx+3, 1):
      if j == idx:
        continue
      pos = (j + 9) % 9
      temp_lst = list(s)
      temp_lst[idx] = s[pos]
      temp_lst[pos] = s[idx]
      ns = ''.join(temp_lst)
      if ns not in mapp:
        mapp.add(ns)
        q.append((ns, step+1))

s = '012345678'
q.append((s, 0))
mapp.add(s)
solve()
\end{lstlisting}

\begin{verbatim}
20
\end{verbatim}

\newpage
    \subsubsection{可行性剪枝\ poj3278}

    在一条直线上, 农夫在N位置, 奶牛在K位置. 农夫要抓到牛, 有3钟移动方法: 若位于X, 可以移动到X-1, X+1, 2X. 农夫需要走多少次才能从N到达K. 

    Input: 两个整数N, K. $0 <= N,\ K <= 100000$

    Output: 最少移动次数

    使用可行性剪枝, 当$X>K$时, 只能移动到X-1, 不能扩大X.

\begin{lstlisting}[style = Python]
from collections import deque

n, k = map(int, input().split())
q = deque()
q.append((n, 0))
rst = []
mapp = set()

def solve():
    global q, n, k, mapp, rst
    while len(q) > 0:
        now = q[0]
        q.popleft()
        x = now[0]
        step = now[1]
        if x == k:
            rst.append(step)
        if x < 0:
            mapp.add(x)
            continue

        if x > k:
            if x-1 not in mapp:
                q.append((x-1, step+1))
                mapp.add(x-1)
        else:
            if x-1 not in mapp:
                q.append((x-1, step+1))
                mapp.add(x-1)
            if x+1 not in mapp:
                q.append((x+1, step+1))
                mapp.add(x+1)
            if 2*x not in mapp:
                q.append((2*x, step+1))
                mapp.add(2*x)
                
            
solve()
rst.sort()
print(rst[0])
\end{lstlisting}

\begin{verbatim}
>> 5 17
<< 4
\end{verbatim}

\newpage
        \subsubsection{最优性剪枝\ 洛谷1118}

            输入一个$1-n$的序列$a_i$, 每次将相邻两个数相加, 形成新的序列, 直到剩下一个数字为止. EX.

            \begin{verbatim}
                3    1    2    4
                   4    3    6
                     7     9
                        16 
            \end{verbatim}

            现在知道n和最终的数字sum, 需要求出最初的序列a. 若答案有多个, 则输出字典序最小的一个.

            \textbf{1.} 从序列计算到最后一个数, 实际上是杨辉三角的应用, 可以先计算出来杨辉三角的系数.

            \textbf{2. 最优性剪枝}: 若当前枚举到的序列中有一段, 这段子序列相加进行上述累加的结果已经超过sum, 则整个序列都可以排除. 

            \textbf{对于Python}: 生成全排列可以使用itertools中的permutations; 列表可以直接进行大小比较, 放回的结果是第一个不同数字的大小比较结果; 查重可以将数字使用空格分隔再转换为字符串进行.

\begin{lstlisting}[style = Python]
from itertools import permutations
from copy import deepcopy

# prepare

cal_tab = dict()
base = [0] * 12
base[0] = 1
cal_tab[1] = base
pass_dict = set()

def move_add(a):
    i = 1
    rst = deepcopy(base)
    while i < len(a):
        if a[i-1] == 0:
            return rst
        rst[i] = a[i-1] + a[i]
        i += 1
    return rst

def tri_sum(a):
    global cal_tab
    n = len(a)
    tab = cal_tab[n]
    rst = 0
    for i in range(n):
        rst += a[i] * tab[i]
    return rst

for i in range(2, 13):
    temp = deepcopy(cal_tab[i-1])
    cal_tab[i] = move_add(temp)

# begin receive inputs
n, summ = map(int, input().split())
origin = [i for i in range(1, n+1)]
rst = []
for per in permutations(origin):
    per = list(per)
    check = ' '.join(map(str, per))
    flag = False
    for item in pass_dict:
        if item in check:
            flag = True
            break
    if flag:
        continue
    if tri_sum(per) == summ:
        if len(rst) == 0:
            rst = per
        elif per < rst:
            rst = per

    elif tri_sum(per) > summ:
        per.pop()
        while len(per) > 0:
            if tri_sum(per) > summ:
                pass_dict.add(' '.join(map(str, per)))
                per.pop()
            else:
                break

print(' '.join(map(str, rst)))
\end{lstlisting}

\begin{verbatim}
>> 4 16
<< 3 1 2 4
\end{verbatim}

\newpage
        \subsubsection{优化搜索顺序\ 排除等效冗余\ 洛谷1120}

        乔治有一些同样长的小木棍,他把这些木棍随意砍成几段,直到每段的长都不超过 50。
        现在,他想把小木棍拼接成原来的样子,但是却忘记了自己开始时有多少根木棍和它们的长度。
        给出每段小木棍的长度,编程帮他找出原始木棍的最小可能长度。

        输入格式

        第一行是一个整数 n,表示小木棍的个数。
        
        第二行有 n 个整数,表示各个木棍的长度$a_i$

        输出格式
        
        输出一行一个整数表示答案。

        \textbf{1. 优化搜索顺序}: 把小木棍按长度从大到小排列, 再按照从大到小的顺序尝试拼接. 

        \textbf{2. 排除等效冗余}: 即按照优化1进行优化, 因为在拼接中, 从大到小地拼接和从小到大的拼接一样.

        \textbf{3. 长度优化}: 所有可行长度D是小木棍总长度的一个约数. 令木棍长度为summ, 则D的范围为[1, summ]. 再令K为原始木棍根数, 则$summ = K\cdot D, K \in Z^+\ \implies D \in \{n | summ/n \in Z^+\}$. 若summ不可分为多个D之和, 则该D不可用于分割.

\begin{lstlisting}[style = Python]
n = int(input())
a = list(map(int, input().split()))
a.sort(reverse=True)
a_mark = [False for i in range(len(a))]
summ = sum(a)

def reset():
    global a_mark
    for i in range(len(a_mark)):
        a_mark[i] = False
def combine(length):
    global a, a_mark
    if length == 0:
        return True
    for i in range(len(a)):
        flag = False
        if a_mark[i]:
            continue
        else:
            if a[i] > length:
                continue
            else:
                a_mark[i] = True
                flag = combine(length - a[i])
                if not flag:
                    a_mark[i] = False
        if flag:
            return True

for length in range(1, summ+1):
    reset()
    flag = True
    if summ % length != 0:
        continue
    else:
        k = summ // length
        for i in range(k):
            flag = combine(length)
            if not flag:
                reset()
                break
    if flag:
        print(length)
        break
\end{lstlisting}

\begin{verbatim}
>> 9
>> 5 2 1 5 2 1 5 2 1
<< 6
\end{verbatim}

\subsection{洪水填充}

从一个种子点开始, 扩散到邻居, 再不断扩散到邻居的邻居的过程. 使用BFS和DFS都可以, DFS更简单.

EX. 

\begin{lstlisting}[style = Python]
import math
from time import sleep

def show(region):
    print()
    for i in range(15):
        for j in range(15):
            print(region[i][j], end=' ')
        print()

region = [[0 for _ in range(15)]for _ in range(15)]
for i in range(15):
    for j in range(15):
        if 4.5 < math.sqrt((i-7)**2+(j-7)**2) < 5.5:
            region[i][j] = 1

print('original')
show(region)
sleep(1)

'''
core
'''
def floodfill(x, y, new_color, old_color):
    global region
    if math.sqrt((x-7)**2+(y-7)**2) < 5.5 and region[y][x] == old_color \
        and 0 <= x <= 14 and 0 <= y <= 14:
        region[y][x] = new_color
        show(region)
        sleep(1)
        floodfill(x+1, y, 1, 0)
        floodfill(x-1, y, 1, 0)
        floodfill(x, y+1, 1, 0)
        floodfill(x, y-1, 1, 0)

floodfill(3, 5, 1, 0)
\end{lstlisting}

\newpage
\subsection{BFS与最短路径}

在所有相邻点的距离\textbf{相等}时, BFS是最优的最短距离算法; 若距离\textbf{不相等}, 需要使用Dijkstra等通用的标准算法.

\textbf{蓝桥 602 改}

给出地图, 1为障碍, 0为可以通行的地方. 迷宫的左上角为入口, 右下角为出口. 找出一种通过迷宫的方式, 其使用的步数最少. 对每一步使用DULR表示, DULR分别为下上左右.

\begin{verbatim}
010000
000100
001001
110000
\end{verbatim}

\begin{lstlisting}[style = Python]
from collections import deque

class node:
    def __init__(self, x, y, path):
        self.x = x
        self.y = y
        self.path = path

rows = 30
cols = 50
a = \
'01010101001011001001010110010110100100001000101010' \
...(略)
'00111100001000010000000110111000000001000000001011' \
'10000001100111010111010001000110111010101101111000'
mapp = [[None for _ in range(cols)]for _ in range(rows)]
for r in range(rows):
    for c in range(cols):
        mapp[r][c] = int(a[c + r * cols])

vis = [[0 for _ in range(cols)]for _ in range(rows)]
op = [[1, 0],
      [-1, 0],
      [0, 1],
      [0, -1]]
P = ['R', 'L', 'D', 'U']

def bfs():
    start = node(0, 0, '')
    vis[0][0] = 1
    q = deque()
    q.append(start)
    while len(q) > 0:
        now = q.popleft()
        if now.x == cols-1 and now.y == rows-1:
            print(now.path)
        # broadcast
        for i in range(4):
            next = node(now.x + op[i][0], now.y + op[i][1], now.path+P[i])
            # out of range
            if next.x < 0 or next.y < 0 or next.x > cols-1 or next.y > rows-1:
                continue
            if vis[next.y][next.x] == 1 or mapp[next.y][next.x] == 1:
                continue
            vis[next.y][next.x] = 1
            q.append(next)

bfs()
\end{lstlisting}





    
    


\end{sloppy}
\end{document}