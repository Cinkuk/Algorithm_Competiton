\documentclass[../main]{subfiles}
\begin{document}
\section{基础数据结构}
    
\subsection{Link List in Python:}

\begin{lstlisting}[style = Python]
class nodes:
    def __init__(self, val=None, pre=None, next=None):
        self.val = val
        self.next = next
\end{lstlisting}

\textbf{EX 1} 


\textbf{洛谷 P1996 约瑟夫问题}

题目描述

n 个人围成一圈,从第一个人开始报数,数到 m 的人出列,再由下一个人重新从 1 开始报数,数到 m 的人再出圈,依次类推,直到所有的人都出圈,请输出依次出圈人的编号。

注意:本题和《深入浅出-基础篇》上例题的表述稍有不同。书上表述是给出淘汰 n−1 名小朋友,而该题是全部出圈。

输入格式

输入两个整数 n,m。

输出格式

输出一行 n 个整数,按顺序输出每个出圈人的编号。

输入输出样例

输入 \#1

10 3

输出 \#1

3 6 9 2 7 1 8 5 10 4

说明/提示

1≤m,n≤100

\begin{lstlisting}[style = Python]
    # 动态链表
    class node:
        data = None
        next = None
    
    n, m = map(int, input().split())
    # init link list
    head = node()
    head.data = 1
    now = head
    for i in range(2, n+1):
        p = node()
        p.data = i
        now.next = p
        now = p
    now.next = head
    prev = now
    now = head
    while n > 0:
        n -= 1
        for i in range(m-1):
            prev = now
            now = now.next
        print(now.data, end=' ')
        prev.next = now.next
        now = now.next
\end{lstlisting}

\newpage
\begin{lstlisting}[style = Python]
    # 列表和索引计算
    n, m = map(int, input().split())
    people = list(range(1, n+1))
    rst = []
    current = 0
    index = 0
    while people:
        index = (current + m - 1) % len(people)
        rst.append(people.pop(index))
        current = index
    print(' '.join(map(str, rst)))
\end{lstlisting}

\textbf{EX 2}


\textbf{洛谷 P1160 队列安排}

题目描述

一个学校里老师要将班上 $N$ 个同学排成一列,同学被编号为 $1\sim N$,他采取如下的方法:

1. 先将 $1$ 号同学安排进队列,这时队列中只有他一个人;

2. $2\sim N$ 号同学依次入列,编号为 $i$ 的同学入列方式为:老师指定编号为 $i$ 的同学站在编号为 $1\sim(i-1)$ 中某位同学(即之前已经入列的同学)的左边或右边;

3. 从队列中去掉 $M$ 个同学,其他同学位置顺序不变。

在所有同学按照上述方法队列排列完毕后,老师想知道从左到右所有同学的编号。

输入格式

第一行一个整数 $N$,表示了有 $N$ 个同学。

第 $2\sim N$ 行,第 $i$ 行包含两个整数 $k,p$,其中 $k$ 为小于 $i$ 的正整数,$p$ 为 $0$ 或者 $1$。若 $p$ 为 $0$,则表示将 $i$ 号同学插入到 $k$ 号同学的左边,$p$ 为 $1$ 则表示插入到右边。

第 $N+1$ 行为一个整数 $M$,表示去掉的同学数目。

接下来 $M$ 行,每行一个正整数 $x$,表示将 $x$ 号同学从队列中移去,如果 $x$ 号同学已经不在队列中则忽略这一条指令。

输出格式

一行,包含最多 $N$ 个空格隔开的整数,表示了队列从左到右所有同学的编号。

输入输出样例 \#1

输入 \#1

4
1 0
2 1
1 0
2
3
3

输出 \#1

2 4 1

说明/提示

**【样例解释】**

将同学 $2$ 插入至同学 $1$ 左边,此时队列为:

2 1

将同学 $3$ 插入至同学 $2$ 右边,此时队列为:

2 3 1  

将同学 $4$ 插入至同学 $1$ 左边,此时队列为:

2 3 4 1  

将同学 $3$ 从队列中移出,此时队列为:

2 4 1

同学 $3$ 已经不在队列中,忽略最后一条指令

最终队列:

2 4 1  

【数据范围】

对于 $20\%$ 的数据,$1\leq N\leq 10$。

对于 $40\%$ 的数据,$1\leq N\leq 1000$。

对于 $100\%$ 的数据,$1<M\leq N\leq 10^5$。


\begin{lstlisting}[style = Python]
    # 链表
    class nodes:
        def __init__(self, val=None, next=None, prev=None):
            self.val = val
            self.next = next
            self.prev = prev
    
    N = int(input())
    people = nodes(1)
    head = people
    idx = {1: people}
    for i in range(2, N+1):
        node = nodes(i)
        k, p = map(int, input().split())
        k = idx[k]
        if p == 0:
            if k.prev:
                k.prev.next = node
                node.prev = k.prev
            node.next = k
            k.prev = node
            if k == head:
                head = node
        elif p == 1:
            if k.next:
                k.next.prev = node
                node.next = k.next
            node.prev = k
            k.next = node
        idx[i] = node
        
    M = int(input())
    for i in range(M):
        x = int(input())
        if x in idx.keys():
            k = idx[x]
            if k.prev:
                k.prev.next = k.next
                if k.next:
                    k.next.prev = k.prev
            if k.next:
                k.next.prev = k.prev
                if k.prev:
                    k.prev.next = k.next
                if k == head:
                    head = k.next
            del idx[x]
    
    while head:
        print(head.val, end=' ')
        head = head.next
\end{lstlisting}

\newpage
\subsection{Queue in Python}

\subsubsection{双端队列}

\begin{lstlisting}[style = Python]
from collections import deque # 双端队列

queue = deque(maxlen = 10) # 最大长度为10, None为无限制
queue = deque(iterable) # init queue by iterable obj.

queue.append(x) # add x to right side
queue.appendleft(x) # add x to left side

queue.extend(iterable) # extend right side by iterable obj.
queue.extendleft(iterable) # extend the left side by iterable obj.

queue.pop() # pop element from right side 
queue.popleft() # pop element from left side

queue.remove(x) # remove x from Queue, raise error if not found
queue.clear() # clear Queue

queue.reverse() # reverse Queue
queue.insert(i, x) # insert x into queue at position i
queue.count(x) # count the number of queue elements = x
queue.index(x) # return first match position, raise error if not found

\end{lstlisting}

\textbf{EX 1}

\textbf{洛谷 P1540 [NOIP 2010 提高组] 机器翻译}

题目背景

NOIP2010 提高组 T1

题目描述

小晨的电脑上安装了一个机器翻译软件,他经常用这个软件来翻译英语文章。

这个翻译软件的原理很简单,它只是从头到尾,依次将每个英文单词用对应的中文含义来替换。对于每个英文单词,软件会先在内存中查找这个单词的中文含义,如果内存中有,软件就会用它进行翻译;如果内存中没有,软件就会在外存中的词典内查找,查出单词的中文含义然后翻译,并将这个单词和译义放入内存,以备后续的查找和翻译。

假设内存中有 $M$ 个单元,每单元能存放一个单词和译义。每当软件将一个新单词存入内存前,如果当前内存中已存入的单词数不超过 $M-1$,软件会将新单词存入一个未使用的内存单元;若内存中已存入 $M$ 个单词,软件会清空最早进入内存的那个单词,腾出单元来,存放新单词。

假设一篇英语文章的长度为 $N$ 个单词。给定这篇待译文章,翻译软件需要去外存查找多少次词典?假设在翻译开始前,内存中没有任何单词。

输入格式

共 $2$ 行。每行中两个数之间用一个空格隔开。

第一行为两个正整数 $M,N$,代表内存容量和文章的长度。

第二行为 $N$ 个非负整数,按照文章的顺序,每个数(大小不超过 $1000$)代表一个英文单词。文章中两个单词是同一个单词,当且仅当它们对应的非负整数相同。

输出格式

一个整数,为软件需要查词典的次数。

输入输出样例 \#1

输入 \#1

3 7
1 2 1 5 4 4 1

输出 \#1

5

说明/提示

样例解释

整个查字典过程如下:每行表示一个单词的翻译,冒号前为本次翻译后的内存状况:

1. 1:查找单词 1 并调入内存。
2. 1 2:查找单词 2 并调入内存。
3. 1 2:在内存中找到单词 1。
4. 1 2 5:查找单词 5 并调入内存。
5. 2 5 4:查找单词 4 并调入内存替代单词 1。
6. 2 5 4:在内存中找到单词 4。
7. 5 4 1:查找单词 1 并调入内存替代单词 2。

共计查了 $5$ 次词典。

数据范围

- 对于 $10\%$ 的数据有 $M=1$,$N \leq 5$;
- 对于 $100\%$ 的数据有 $1 \leq M \leq 100$,$1 \leq N \leq 1000$。


\begin{lstlisting}[style = Python]
# 队列 collections.deque
from collections import deque
from array import array

M, N = map(int, input().split())
q = deque(maxlen=M) # 双向队列
qs = set() # 用作哈希表
count = 0 # 计数
inp_ary = array('i', map(int, input().split())) # 输入数据
l = len(inp_ary)
for i in range(l):
    if inp_ary[i] not in qs:
        if len(q) == M:
            qs.remove(q.popleft())
        qs.add(inp_ary[i])
        q.append(inp_ary[i])
        count += 1
print(count)
\end{lstlisting}

\subsubsection{单调队列}

\textbf{洛谷 P1886 滑动窗口 /【模板】单调队列}

题目描述

有一个长为 $n$ 的序列 $a$,以及一个大小为 $k$ 的窗口。现在这个从左边开始向右滑动,每次滑动一个单位,求出每次滑动后窗口中的最大值和最小值。

例如,对于序列 $[1,3,-1,-3,5,3,6,7]$ 以及 $k = 3$,有如下过程:

$$\def\arraystretch{1.2}
\begin{array}{|c|c|c|}\hline
\textsf{窗口位置} & \textsf{最小值} & \textsf{最大值} \\ \hline
\verb![1   3  -1] -3   5   3   6   7 ! & -1 & 3 \\ \hline
\verb! 1  [3  -1  -3]  5   3   6   7 ! & -3 & 3 \\ \hline
\verb! 1   3 [-1  -3   5]  3   6   7 ! & -3 & 5 \\ \hline
\verb! 1   3  -1 [-3   5   3]  6   7 ! & -3 & 5 \\ \hline
\verb! 1   3  -1  -3  [5   3   6]  7 ! & 3 & 6 \\ \hline
\verb! 1   3  -1  -3   5  [3   6   7]! & 3 & 7 \\ \hline
\end{array}
$$

输入格式

输入一共有两行,第一行有两个正整数 $n,k$。
第二行 $n$ 个整数,表示序列 $a$

输出格式

输出共两行,第一行为每次窗口滑动的最小值   
第二行为每次窗口滑动的最大值

输入输出样例 \#1

输入 \#1

8 3
1 3 -1 -3 5 3 6 7

输出 \#1

-1 -3 -3 -3 3 3
3 3 5 5 6 7

说明/提示

【数据范围】    
对于 $50\%$ 的数据,$1 \le n \le 10^5$;  
对于 $100\%$ 的数据,$1\le k \le n \le 10^6$,$a_i \in [-2^{31},2^{31})$。

\begin{lstlisting}[style = Python]
from collections import deque
from array import array
import sys

# input 
n, k = map(int, input().split())
a = list(map(int, input().split()))

min_q = deque()
max_q = deque()
min_rst = []
max_rst = []

for i in range(n):
    # minimum queue
    while min_q and a[min_q[-1]] > a[i]:
        min_q.pop()
    min_q.append(i)
    # maximum queue
    while max_q and a[max_q[-1]] < a[i]:
        max_q.pop()
    max_q.append(i)

    # 开始记录
    if i >= k - 1:
        min_rst.append(a[min_q[0]])
        max_rst.append(a[max_q[0]])
    
    # 弹出窗口外的元素
    if min_q and min_q[0] <= i - k + 1:
        min_q.popleft()
    if max_q and max_q[0] <= i - k + 1:
        max_q.popleft()

# output
print(' '.join(map(str, min_rst)))
print(' '.join(map(str, max_rst)))
\end{lstlisting}

\subsubsection{单调队列与动态规划}

\textbf{通过前缀和 + 单调队列 + 动态规划求解子序和问题}

洛谷 P1714

Problem Description

\textbf{给定一个序列, 给定一个最大长度m, 求一段长度不超过m的连续子序列, 使其子序和最大}
 

Input

第1行输入n, m. 分别为序列长度和最大长度

第2行输入N个数

Output

第1个数是最大子序和, 第2和第3个数是开始和终止位置
 
Case 1:

Input

5 2

1 2 3 4 5

Output

9

Case 2:

Input

6 3

1 -2 3 -4 5 -6
 
Output

5

Case 3

Input

5 5

1 2 3 4 5

Output

15

\begin{lstlisting}[style = Python]
# 洛谷 P1714

from collections import deque
n, m = map(int, input().split())
s = list(map(int, input().split())) # 前缀和
dp = deque()
ans = s[0]

# 计算前缀和
for i in range(1, n):
    s[i] = s[i-1] + s[i]
# 单调队列
dp.append(0)
for i in range(1, n):
    # 去头
    while len(dp) > 0 and dp[0] < i- m:
        dp.popleft()
    # 去尾, 去除>=s[i]的元素, 使得s[j] - s[i]更大
    while len(dp) > 0 and s[dp[-1]] >= s[i]:
        dp.pop()
    dp.append(i)
    if len(dp) == n:
        ans = max(ans, s[i])
    elif len(dp) > 1:
        ans = max(ans, s[i] - s[dp[0]])
    elif len(dp) == 1:
        ans = max(ans, s[dp[0]])
    else:
        ans = max(ans, s[i])
    
print(ans)
\end{lstlisting}

\newpage
\subsubsection{优先队列}

\begin{lstlisting}[style = Python]
from queue import PriorityQueue
MAXSiZE = 0

q = PriorityQueue(maxsize=MAXSiZE) # 队列大小, 默认为0, <=0的队列大小为无穷
q1 = PriorityQueue()

q.empty() # if empty
q.full() # if full
q.qsize() # get current size

# add elem
# 两种添加方式不能混用
# 直接按照元素大小存入, 元素大小越小, 优先级越高
q.put(1) 
q.put(2)
# q.put((priority number, data))
# priority number越小, 优先级越高
q1.put((2, 'ele'))
q1.put((-3, 'ele2'))

# get element, = pop()
q.get()
\end{lstlisting}

\subsection{Stack in Python}

\textbf{使用deque作为栈.}

\indent\par

hdu 1062

\textbf{翻转字符串}

input:

olleh !dlrow

output:

hello world!

\begin{lstlisting}[style = Python]
from collections import deque
    s = deque()
    inp = list(input().split())
    for item in inp:
        # enter stack
        for i in item:
            s.append(i)
        while len(s) > 0:
            print(s.pop(), end='')
        print(' ', end='')
\end{lstlisting}

\newpage
\subsubsection{单调栈}

\textbf{洛谷 P2947 [USACO09MAR] Look Up S}

题目描述

Farmer John's N (1 <= N <= 100,000) cows, conveniently numbered 1..N, are once again standing in a row. Cow i has height H\_i (1 <= H\_i <= 1,000,000).

Each cow is looking to her left toward those with higher index numbers. We say that cow i 'looks up' to cow j if i < j and H\_i < H\_j. For each cow i, FJ would like to know the index of the first cow in line looked up to by cow i.

Note: about 50% of the test data will have N <= 1,000. 

约翰的 $N(1\le N\le10^5)$ 头奶牛站成一排,奶牛 $i$ 的身高是 $H_i(1\le H_i\le10^6)$。现在,每只奶牛都在向右看齐。对于奶牛 $i$,如果奶牛 $j$ 满足 $i<j$ 且 $H_i<H_j$,我们可以说奶牛 $i$ 可以仰望奶牛 $j$。 求出每只奶牛离她最近的仰望对象。

Input

输入格式

1. \* Line 1: A single integer: N

\* Lines 2..N+1: Line i+1 contains the single integer: H\_i

第 $1$ 行输入 $N$,之后每行输入一个身高 $H_i$。

输出格式

\* Lines 1..N: Line i contains a single integer representing the smallest index of a cow up to which cow i looks. If no such cow exists, print 0.

共 $N$ 行,按顺序每行输出一只奶牛的最近仰望对象,如果没有仰望对象,输出 $0$。

输入输出样例 \# 1

输入 \# 1

6 

3 

2 

6 

1 

1 

2

输出 \# 1

3 

3 

0 

6 

6 

0

说明/提示

FJ has six cows of heights 3, 2, 6, 1, 1, and 2.


Cows 1 and 2 both look up to cow 3; cows 4 and 5 both look up to cow 6; and cows 3 and 6 do not look up to any cow.

【输入说明】$6$ 头奶牛的身高分别为 3,2,6,1,1,2。

【输出说明】奶牛 \#1,\#2 仰望奶牛 \#3,奶牛 \#4,\#5 仰望奶牛 \#6,奶牛 \#3 和 \#6 没有仰望对象。

【数据规模】

对于 $20\%$ 的数据:$1\le N\le10$;

对于 $50\%$ 的数据:$1\le N\le10^3$;

对于 $100\%$ 的数据:$1\le N\le10^5,1\le H_i\le10^6$。

\begin{lstlisting}[style = Python]
from collections import deque
from array import array
n = int(input())
cows = []
for i in range(n):
    cows.append(int(input()))
s = deque()
rst = array('i')
for i in range(n-1, -1, -1):
    # 由于反向遍历, 栈内元素必然在当前遍历元素的右边
    # 不比当前高的元素出栈, 保证栈底到栈顶的高度递减
    # 目的是使得随着逐渐出栈, 栈顶元素逐渐变高, 以找到能比当前遍历元素高的元素
    while len(s) > 0 and cows[s[-1]] <= cows[i]:
        s.pop()
    if len(s) == 0:
        rst.append(0)
    else:
        rst.append(s[-1] + 1) # relationship between index & NO.
    s.append(i)
for i in range(len(rst)-1, -1, -1):
    print(rst[i])
\end{lstlisting}

\subsection{Binary Tree in Python}

\begin{lstlisting}[style = Python]
class nodes:
    def __init__(self, val=None, l=None, r=None):
        self.val = val
        self.l = l
        self.r = r
\end{lstlisting}

\subsubsection{DFS遍历}

\begin{lstlisting}[style = Python]
# pre
def preorder(node):
    if not node:
        return
    print(node.val, end=' ')
    preorder(node.l)
    preorder(node.r)

# mid
def inorder(node):
    if not node:
        return
    inorder(node.l)
    print(node.val, end=' ')
    inorder(node.r)

# post 
def postorder(node):
    if not node:
        return
    postorder(node.l)
    postorder(node.r)
    print(node.val, end=' ')
\end{lstlisting}

\newpage
\begin{lstlisting}[style = Python]
# pre
def preorder(node):
    if not node:
        return
    print(node.val, end=' ')
    preorder(node.l)
    preorder(node.r)

# mid
def inorder(node):
    if not node:
        return
    inorder(node.l)
    print(node.val, end=' ')
    inorder(node.r)

# post 
def postorder(node):
    if not node:
        return
    postorder(node.l)
    postorder(node.r)
    print(node.val, end=' ')
\end{lstlisting}

\begin{lstlisting}[style = Python]
# test code
ab = ['A', 'B', 'C', 'D', 'E', 'F', 'G', 'H', 'I']
n = list(nodes(i) for i in ab)
root = n[4]
root.l = n[1]
root.r = n[6]
root.l.l = n[0]
root.l.r = n[3]
root.l.r.l = n[2]
root.r.l = n[5]
root.r.r = n[-1]
root.r.r.l = n[-2]
\end{lstlisting}

\begin{lstlisting}[style = Pseudocode]
// output
E B A D C G F I H 
A B C D E F G H I 
A C D B F H I G E 
\end{lstlisting}

\newpage
\textbf{根据前序和中序遍历输出后序遍历结果}
\begin{lstlisting}[style = Python]
n = int(input())
global pre_tra
pre_tra = list(map(int, input().split()))
in_tra = list(map(int, input().split()))

root = nodes(pre_tra[0])
pre_tra = pre_tra[1:]
# split left and right subtree
mid = in_tra.index(root.val)
lp = in_tra[0:mid]
rp = in_tra[mid+1:]

def establish(root, lpart, rpart):
    global pre_tra
    # left part
    if lpart and pre_tra:
        # connect subtree
        pivot = nodes(pre_tra[0])
        pre_tra = pre_tra[1:]
        root.l = pivot
        # split left and right subtree
        mid = lpart.index(pivot.val)
        lp = lpart[0:mid]
        rp = lpart[mid+1:]
        establish(pivot, lp, rp)
    # right part
    if rpart and pre_tra:
        # connect subtree
        pivot = nodes(pre_tra[0])
        pre_tra = pre_tra[1:]
        root.r = pivot
        # split left and right subtree
        mid = rpart.index(pivot.val)
        lp = rpart[0:mid]
        rp = rpart[mid+1:]
        establish(pivot, lp, rp)

establish(root, lp, rp)
postorder(root)
\end{lstlisting}

\subsubsection{哈夫曼编码}

poj 1521

Question:
\textbf{输入一个字符串, 分别输出ASCII(8bit / character)和哈夫曼编码的长度, 以及压缩比}

Sample Input

AAAAABCD

THE\_CAT\_IN\_THE\_HAT

END

Sample Output

64 13 4.9

144 51 2.8

\newpage
\begin{lstlisting}[style = Python]
from queue import PriorityQueue
s = input()
while s != 'END':
    chaSet = set(item for item in s)
    q = PriorityQueue()

    # 只有一个字符的情况, 建哈夫曼树至少需要两个节点
    if len(s) == 1:
        print('8 1 8')
    else:
        for item in chaSet:
            q.put(s.count(item))
        rst = 0
        while q.qsize() > 1:
            a = q.get()
            b = q.get()
            q.put(a + b)
            # 每增加一层, 编码长度加1
            rst += a + b
        
        # clear queue
        q.get()
        print('{} {} {}'.format(str(8 * len(s)),
                                str(rst),
                                str(8 * len(s) / rst)))

    s = input()
\end{lstlisting}

\subsection{Heap in Python}

\begin{lstlisting}[style = Python]
import heapq

# heapq创建的是小根堆, 通过对元素取负可以转换为大根堆

# create a heap
# convert a list to heap
lst = [2, 8, 1, 63, 8, 1, 0, 4]
heapq.heapify(lst)
# 此后堆heap的任何操作都要通过库函数操作
print(lst)

# heapq.heappush(heap, item)
heapq.heappush(lst, 5)
print(lst)

# heapq.heappop(heap)
# -> pop and return the min element of heap
print(heapq.heappop(lst))
print(lst)

# heapq.heappushpop(heap, item)
# -> push item into heap, and return the min element
print(heapq.heappushpop(lst, 4))
print(lst)
\end{lstlisting}

\textbf{P3378 【模板】堆}

题目描述

给定一个数列,初始为空,请支持下面三种操作:

1. 给定一个整数 $x$,请将 $x$ 加入到数列中。

2. 输出数列中最小的数。

3. 删除数列中最小的数(如果有多个数最小,只删除 $1$ 个)。

输入格式

第一行是一个整数,表示操作的次数 $n$。  

接下来 $n$ 行,每行表示一次操作。每行首先有一个整数 $op$ 表示操作类型。

- 若 $op = 1$,则后面有一个整数 $x$,表示要将 $x$ 加入数列。

- 若 $op = 2$,则表示要求输出数列中的最小数。

- 若 $op = 3$,则表示删除数列中的最小数。如果有多个数最小,只删除 $1$ 个。

输出格式

对于每个操作 $2$,输出一行一个整数表示答案。

输入输出样例 1

输入 1

5

1 2

1 5

2

3

2

输出 1

2

5

说明/提示

【数据规模与约定】

- 对于 $30\%$ 的数据,保证 $n \leq 15$。

- 对于 $70\%$ 的数据,保证 $n \leq 10^4$。

- 对于 $100\%$ 的数据,保证 $1 \leq n \leq 10^6$,$1 \leq x < 2^{31}$,$op \in \{1, 2, 3\}$。


\begin{lstlisting}[style = Python]
import heapq
lt = list()
heapq.heapify(lt)
n = int(input())
for i in range(n):
    op = list(map(int, input().split()))
    if len(op) == 2:
        heapq.heappush(lt, op[1])
    elif len(op) == 1:
        if op[0] == 2:
            rst = heapq.nsmallest(1, lt)
            print(rst[0])
        elif op[0] == 3:
            heapq.heappop(lt)
\end{lstlisting}




\end{document}