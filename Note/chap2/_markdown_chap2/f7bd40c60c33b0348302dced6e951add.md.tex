\markdownRendererDocumentBegin
\markdownRendererWarning{The `hybrid` option has been soft-deprecated.}{The `hybrid` option has been soft-deprecated.}{Consider using one of the following better options for mixing TeX and markdown: `contentBlocks`, `rawAttribute`, `texComments`, `texMathDollars`, `texMathSingleBackslash`, and `texMathDoubleBackslash`. For more information, see the user manual at <https://witiko.github.io/markdown/>.}{Consider using one of the following better options for mixing TeX and markdown: `contentBlocks`, `rawAttribute`, `texComments`, `texMathDollars`, `texMathSingleBackslash`, and `texMathDoubleBackslash`. For more information, see the user manual at <https://witiko.github.io/markdown/>.}\markdownRendererSectionBegin
\markdownRendererHeadingOne{P1093 [NOIP 2007 普及组] 奖学金}\markdownRendererInterblockSeparator
{}\markdownRendererSectionBegin
\markdownRendererHeadingTwo{题目背景}\markdownRendererInterblockSeparator
{}NOIP2007 普及组 T1\markdownRendererInterblockSeparator
{}
\markdownRendererSectionEnd \markdownRendererSectionBegin
\markdownRendererHeadingTwo{题目描述}\markdownRendererInterblockSeparator
{}某小学最近得到了一笔赞助,打算拿出其中一部分为学习成绩优秀的前 $5$ 名学生发奖学金。期末,每个学生都有 $3$ 门课的成绩:语文、数学、英语。先按总分从高到低排序,如果两个同学总分相同,再按语文成绩从高到低排序,如果两个同学总分和语文成绩都相同,那么规定学号小的同学排在前面,这样,每个学生的排序是唯一确定的。\markdownRendererParagraphSeparator
{}任务:先根据输入的 $3$ 门课的成绩计算总分,然后按上述规则排序,最后按排名顺序输出前五名名学生的学号和总分。\markdownRendererParagraphSeparator
{}注意,在前 $5$ 名同学中,每个人的奖学金都不相同,因此,你必须严格按上述规则排序。例如,在某个正确答案中,如果前两行的输出数据(每行输出两个数:学号、总分) 是:\markdownRendererInterblockSeparator
{}\markdownRendererInputFencedCode{./_markdown_chap2/d99bd9958be6049ee371952ceb7895ee.verbatim}{plain}{plain}\markdownRendererInterblockSeparator
{}这两行数据的含义是:总分最高的两个同学的学号依次是 $7$ 号、$5$ 号。这两名同学的总分都是 $279$ (总分等于输入的语文、数学、英语三科成绩之和) ,但学号为 $7$ 的学生语文成绩更高一些。\markdownRendererParagraphSeparator
{}如果你的前两名的输出数据是:\markdownRendererInterblockSeparator
{}\markdownRendererInputFencedCode{./_markdown_chap2/cde3af711afb708b8f2b85a836c74b8b.verbatim}{plain}{plain}\markdownRendererInterblockSeparator
{}则按输出错误处理,不能得分。\markdownRendererInterblockSeparator
{}
\markdownRendererSectionEnd \markdownRendererSectionBegin
\markdownRendererHeadingTwo{输入格式}\markdownRendererInterblockSeparator
{}共 $n+1$ 行。\markdownRendererParagraphSeparator
{}第 $1$ 行为一个正整数 $n \le 300$,表示该校参加评选的学生人数。\markdownRendererParagraphSeparator
{}第 $2$ 到 $n+1$ 行,每行有 $3$ 个用空格隔开的数字,每个数字都在 $0$ 到 $100$ 之间。第 $j$ 行的 $3$ 个数字依次表示学号为 $j-1$ 的学生的语文、数学、英语的成绩。每个学生的学号按照输入顺序编号为 $1\sim n$(恰好是输入数据的行号减 $1$)。\markdownRendererParagraphSeparator
{}保证所给的数据都是正确的,不必检验。\markdownRendererInterblockSeparator
{}
\markdownRendererSectionEnd \markdownRendererSectionBegin
\markdownRendererHeadingTwo{输出格式}\markdownRendererInterblockSeparator
{}共 $5$ 行,每行是两个用空格隔开的正整数,依次表示前 $5$ 名学生的学号和总分。\markdownRendererInterblockSeparator
{}
\markdownRendererSectionEnd \markdownRendererSectionBegin
\markdownRendererHeadingTwo{输入输出样例 #1}\markdownRendererInterblockSeparator
{}\markdownRendererSectionBegin
\markdownRendererHeadingThree{输入 #1}\markdownRendererInterblockSeparator
{}\markdownRendererInputFencedCode{./_markdown_chap2/3b4385b098fe244634574ee55950771c.verbatim}{}{}\markdownRendererInterblockSeparator
{}
\markdownRendererSectionEnd \markdownRendererSectionBegin
\markdownRendererHeadingThree{输出 #1}\markdownRendererInterblockSeparator
{}\markdownRendererInputFencedCode{./_markdown_chap2/9aade1bfd1189627aebb74f2c1219e19.verbatim}{}{}\markdownRendererInterblockSeparator
{}
\markdownRendererSectionEnd 
\markdownRendererSectionEnd \markdownRendererSectionBegin
\markdownRendererHeadingTwo{输入输出样例 #2}\markdownRendererInterblockSeparator
{}\markdownRendererSectionBegin
\markdownRendererHeadingThree{输入 #2}\markdownRendererInterblockSeparator
{}\markdownRendererInputFencedCode{./_markdown_chap2/29c99572ee91dd711d8f052f17b5ebea.verbatim}{}{}\markdownRendererInterblockSeparator
{}
\markdownRendererSectionEnd \markdownRendererSectionBegin
\markdownRendererHeadingThree{输出 #2}\markdownRendererInterblockSeparator
{}\markdownRendererInputFencedCode{./_markdown_chap2/9f18b1a077b12060b0b5878b5ddcce35.verbatim}{}{}
\markdownRendererSectionEnd 
\markdownRendererSectionEnd 
\markdownRendererSectionEnd \markdownRendererDocumentEnd