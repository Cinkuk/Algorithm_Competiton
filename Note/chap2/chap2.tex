\documentclass[../main]{subfiles}

\begin{document}
\begin{sloppy}
\section{基本算法}
\subsection{尺取法(双指针)}

用以解决序列的区间问题, 一般有两个要求:

1. \textbf{序列是有序的, 需要先对序列进行排序}

2. 问题与序列的区间有关, 操作两个或多个指针i, j表示区间

在Python中, 用while实现较为方便

\textbf{扫描方向}: 

\textbf{反向扫描, 左右指针}: i, j的方向相反; 

\textbf{同向扫描, 快慢指针}: i, j的方向相同, 但是扫描速度一般不同, 可以形成一个大小可变的滑动窗口

\subsubsection{反向扫描}

\textbf{找指定和的整数对}

问题: 输入n $(n <= 100000)$ 个整数, 放在数组a[]中. 找出其中的两个数, 它们之和等于整数m. (假定肯定有解).

输入:

第1行是数组a[], 第2行是m

Sample input

21 4 5 6 13 65 32 9 23

28

Sample output

5 23

\begin{lstlisting}[style = Python]
# 哈希 复杂度为O(n), 但是需要较大的哈希空间
a = list(map(int, input().split()))
m = int(input())
s = set(a)
outed = set()
for item in s:
    if m - item in s and item not in outed:
        print(item, m - item)
        outed.add(m-item)
\end{lstlisting}

\begin{lstlisting}[style = Python]
# 尺取法 复杂度为O(n log_2^n), 其中, 排序的复杂度为O(log_2^n), 检查的复杂度为O(n)
a = list(map(int, input().split()))
a.sort()
m = int(input())

# 双指针
i, j = 0, len(a) - 1
while (i < j):
    s = a[i] + a[j]
    # s < m: i增加1, 之后的s>=当前s
    if s < m:
        i += 1
    elif s > m:
        j -= 1
    else:
        print("{} {}".format(a[i], a[j]))
        i += 1
\end{lstlisting}

\newpage
\textbf{判断回文串}

输入: 第1行输入测试实例个数, 之后每行输入一个字符串

输出: 是回文串输出yes, 不是输出no

\begin{lstlisting}[style = Python]
n = int(input())
for i in range(n):
    s = str(input())
    i, j = 0, len(s) - 1
    while i < j:
        flag = False
        if s[i] == s[j]:
            flag = True
        else:
            flag = False
            break
        i += 1
        j -= 1
    if (flag):
        print('yes')
    else:
        print('no')
\end{lstlisting}

\subsubsection{同向扫描}

\textbf{使用尺取法产生滑动窗口}

\textbf{寻找区间和}

给定一个长度为n的正整数数组a[]和一个数s, 在数组中找一个区间, 使得该区间的数组元素之和等于s. 输出区间的起点和终点位置

第1行输入数组长度n, 第2行输入数组, 第3行为s

Sample input

15

6 1 2 3 4 6 4 2 8 9 10 11 12 13 14

6

Sample output

0 0

1 3

5 5

6 7

初始值i = j = 0

如果sum = s: 输出一个解, sum减去a[i], i++

如果sum < s: j++, sum + a[j]

如果sum > s: sum - a[i], i++

\newpage
\begin{lstlisting}[style = Python]
n = int(input())
a = list(map(int, input().split()))
s = int(input())

sum = a[0]
i, j = 0, 0
while i < n and j < n:
    if sum == s:
        print("{} {}".format(i, j))
        sum -= a[i]
        i += 1
        j += 1
        sum += a[j]
    elif sum < s:
        j += 1
        sum += a[j]
    elif sum > s:
        sum -= a[i]
        i += 1
\end{lstlisting}

\textbf{数组去重}

给出一个数组, 输出去除重复元素之后的数组

\begin{lstlisting}[style = Python]
# 哈希, 数据多或者数值过大时需要占用大量的空间
a = list(map(int, input().split()))
s = set(a)
unique_a = list(s)
print(unique_a)
\end{lstlisting}

\begin{lstlisting}[style = Python]
# 尺取法
a = list(map(int, input().split()))
# a排序, 使得相同元素排列在一起
a.sort()
n = len(a)
# 双指针均从0开始
i, j = 0, 0
# j始终指向无重复元素部分的最后一个元素
while i < n and j < n:
    # 若i和j指向的元素不同, j++, 将i指向的元素复制到j上
    # EX: 1 2(j) 3(i) 3
    # ->  1 2 3(j, i) 3
    # EX: 1 2 3(j) 3 4(i) 
    # ->  1 2 3 3(j) 4(i)
    # ->  1 2 3 4(j) 4(i)
    if a[i] != a[j]:
        j += 1
        a[j] = a[i]
    i += 1
unique_a = a[0:j+1]
print(unique_a)
\end{lstlisting}

\newpage
\textbf{找相同数对}

给出一串数字和一个数字C, 要求计算出所有A - B = C的数对的个数(不同位置的数字一样的数对算不同的数对)

输入: 2行, 第1行输入整数n和C, 第2行输入n个整数

输出: 满足A - B = C的数对的个数

Sample Input 

6 3

8 4 5 7 7 4

Sample Output

5

\begin{lstlisting}[style = Python]
n, c = map(int, input().split())
a = list(map(int, input().split()))
a.sort()

i, j, k = 0, 0, 0
ans = 0

for i in range(n):
    # j, k指向相同元素区间的起点和终点后1个元素
    # 寻找的对象是区间内的元素 - a[i] = C
    while j < n - 1 and a[j] - a[i] < c:
        j += 1
    while k < n and a[k] - a[i] <= c:
        k += 1
    if a[j] - a[i] == c and a[k-1] - a[i] == c and k - 1 >= 0:
        ans += k - j
print(ans)
\end{lstlisting}

\end{sloppy}
\end{document}