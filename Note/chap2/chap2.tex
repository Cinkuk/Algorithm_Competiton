\documentclass[../main]{subfiles}

\begin{document}
\begin{sloppy}
\section{基本算法}
\subsection{尺取法(双指针)}

用以解决序列的区间问题, 一般有两个要求:

1. \textbf{序列是有序的, 需要先对序列进行排序}

2. 问题与序列的区间有关, 操作两个或多个指针i, j表示区间

在Python中, 用while实现较为方便

\textbf{扫描方向}: 

\textbf{反向扫描, 左右指针}: i, j的方向相反; 

\textbf{同向扫描, 快慢指针}: i, j的方向相同, 但是扫描速度一般不同, 可以形成一个大小可变的滑动窗口

\subsubsection{反向扫描}

\textbf{找指定和的整数对}

问题: 输入n $(n <= 100000)$ 个整数, 放在数组a[]中. 找出其中的两个数, 它们之和等于整数m. (假定肯定有解).

输入:

第1行是数组a[], 第2行是m

Sample input

21 4 5 6 13 65 32 9 23

28

Sample output

5 23

\begin{lstlisting}[style = Python]
# 哈希 复杂度为O(n), 但是需要较大的哈希空间
a = list(map(int, input().split()))
m = int(input())
s = set(a)
outed = set()
for item in s:
    if m - item in s and item not in outed:
        print(item, m - item)
        outed.add(m-item)
\end{lstlisting}

\begin{lstlisting}[style = Python]
# 尺取法 复杂度为O(n log_2^n), 其中, 排序的复杂度为O(log_2^n), 检查的复杂度为O(n)
a = list(map(int, input().split()))
a.sort()
m = int(input())

# 双指针
i, j = 0, len(a) - 1
while (i < j):
    s = a[i] + a[j]
    # s < m: i增加1, 之后的s>=当前s
    if s < m:
        i += 1
    elif s > m:
        j -= 1
    else:
        print("{} {}".format(a[i], a[j]))
        i += 1
\end{lstlisting}

\newpage
\textbf{判断回文串}

输入: 第1行输入测试实例个数, 之后每行输入一个字符串

输出: 是回文串输出yes, 不是输出no

\begin{lstlisting}[style = Python]
n = int(input())
for i in range(n):
    s = str(input())
    i, j = 0, len(s) - 1
    while i < j:
        flag = False
        if s[i] == s[j]:
            flag = True
        else:
            flag = False
            break
        i += 1
        j -= 1
    if (flag):
        print('yes')
    else:
        print('no')
\end{lstlisting}

\subsubsection{同向扫描}

\textbf{使用尺取法产生滑动窗口}

\textbf{寻找区间和}

给定一个长度为n的正整数数组a[]和一个数s, 在数组中找一个区间, 使得该区间的数组元素之和等于s. 输出区间的起点和终点位置

第1行输入数组长度n, 第2行输入数组, 第3行为s

Sample input

15

6 1 2 3 4 6 4 2 8 9 10 11 12 13 14

6

Sample output

0 0

1 3

5 5

6 7

初始值i = j = 0

如果sum = s: 输出一个解, sum减去a[i], i++

如果sum < s: j++, sum + a[j]

如果sum > s: sum - a[i], i++

\newpage
\begin{lstlisting}[style = Python]
n = int(input())
a = list(map(int, input().split()))
s = int(input())

sum = a[0]
i, j = 0, 0
while i < n and j < n:
    if sum == s:
        print("{} {}".format(i, j))
        sum -= a[i]
        i += 1
        j += 1
        sum += a[j]
    elif sum < s:
        j += 1
        sum += a[j]
    elif sum > s:
        sum -= a[i]
        i += 1
\end{lstlisting}

\textbf{数组去重}

给出一个数组, 输出去除重复元素之后的数组

\begin{lstlisting}[style = Python]
# 哈希, 数据多或者数值过大时需要占用大量的空间
a = list(map(int, input().split()))
s = set(a)
unique_a = list(s)
print(unique_a)
\end{lstlisting}

\begin{lstlisting}[style = Python]
# 尺取法
a = list(map(int, input().split()))
# a排序, 使得相同元素排列在一起
a.sort()
n = len(a)
# 双指针均从0开始
i, j = 0, 0
# j始终指向无重复元素部分的最后一个元素
while i < n and j < n:
    # 若i和j指向的元素不同, j++, 将i指向的元素复制到j上
    # EX: 1 2(j) 3(i) 3
    # ->  1 2 3(j, i) 3
    # EX: 1 2 3(j) 3 4(i) 
    # ->  1 2 3 3(j) 4(i)
    # ->  1 2 3 4(j) 4(i)
    if a[i] != a[j]:
        j += 1
        a[j] = a[i]
    i += 1
unique_a = a[0:j+1]
print(unique_a)
\end{lstlisting}

\newpage
\textbf{找相同数对}

洛谷 P1102

给出一串数字和一个数字C, 要求计算出所有A - B = C的数对的个数(不同位置的数字一样的数对算不同的数对)

输入: 2行, 第1行输入整数n和C, 第2行输入n个整数

输出: 满足A - B = C的数对的个数

Sample Input 

6 3

8 4 5 7 7 4

Sample Output

5

\begin{lstlisting}[style = Python]
n, c = map(int, input().split())
a = list(map(int, input().split()))
a.sort()

i, j, k = 0, 0, 0
ans = 0

for i in range(n):
    # j, k指向相同元素区间的起点和终点后1个元素
    # 寻找的对象是区间内的元素 - a[i] = C
    while j < n - 1 and a[j] - a[i] < c:
        j += 1
    while k < n and a[k] - a[i] <= c:
        k += 1
    if a[j] - a[i] == c and a[k-1] - a[i] == c and k - 1 >= 0:
        ans += k - j
print(ans)
\end{lstlisting}

\newpage

\subsection{二分法}

\subsubsection{Python二分搜索库\ bisect}

\begin{lstlisting}[style = Python]
from bisect import *
def fun(find, x, bias=0):
    global a
    index = find(a, x) + bias
    print("index: {}, element: {}".format(index, a[index]))

global a
a = [1, 2, 4, 4, 4, 5]
# target: x
x = 4

# first > x
fun(bisect_right, x)
# first >= x
fun(bisect_left, x)
# first = x
fun(bisect_left, x)
# last = x
fun(bisect_right, x, bias=-1)
# last <= x
fun(bisect_right, 3, bias=-1)
# last < x
fun(bisect_left, x, bias=-1)
# count x in a monotonic array
# slow
print(a.count(x))
# fast, using binary search
print(bisect_right(a, x) - bisect_left(a, x))
\end{lstlisting}

\textbf{output}

\begin{lstlisting}[style = Pseudocode]
index: 5, element: 5
index: 2, element: 4
index: 2, element: 4
index: 4, element: 4
index: 1, element: 2
index: 1, element: 2
3
3
\end{lstlisting}

\subsubsection{整数二分}

\textbf{需要注意终止边界和左右区间问题, 避免漏解和死循环}

\indent\par

\textbf{mid的计算}

\begin{lstlisting}[style = Python]
# 适用单调递增序列的后继问题
mid = l + (r - l) // 2 # 相当于计算出的mid向下取整, 计算的是左中位数
# 适用单调递增序列的前驱问题
mid = l + (r - l + 1) // 2 # 相当于计算出的mid向上取整, 计算的是右中位数
\end{lstlisting}

\newpage

\textbf{在单调递增序列中寻找x或x的后继}

在单调递增序列中寻找第一个x的位置, 若没有x, 则寻找比x大的第一个数的位置, 即寻找第一个$
>= x$的位置

\begin{lstlisting}[style = Python]
# 左闭右开[0, n)
l, r = 0, n
while l < r:
    mid = l + (r - l) // 2
    if a[mid] >= x:
        r = mid
    else:
        l = mid + 1
return l
\end{lstlisting}

\textbf{在单调递增序列中寻找x或x的前驱}

在单调递增序列中寻找第一个x的位置, 若没有x, 则寻找比x小的第一个数的位置, 即寻找第一个$<=x$的位置

\begin{lstlisting}[style = Python]
# 左开右闭(-1, n-1]
l, r = -1, n - 1
while l < r:
    mid = l + (r - l + 1) // 2
    if a[mid] <= x:
        l = mid
    else:
        r = mid - 1
return l
\end{lstlisting}

\textbf{寻找minimum}

\begin{lstlisting}[style = Python]
while l < r:
mid = l + (l - r) // 2
if check(mid):
    # reduce
    r = mid
else:
    # enlarge
    l = mid + 1
# 若r = mid - 1, 则当best=mid时可能遗漏最优解
\end{lstlisting}

\newpage

\textbf{寻找指定和的整数对}

输入n ($n <= 100000$) 个整数, 找出其中的两个数, 使它们之和等于整数m, 假设肯定有解

\begin{lstlisting}[style = Python]
from bisect import *

a = list(map(int, input().split()))
m = int(input())
n = len(a)
a.sort()

# ver. 1
for i in range(n-1):
    # bisearch, a[k] = m - a[i]
    l, r = i+1, n
    x = m - a[i]
    while l < r:
        mid = l + (r - l) // 2
        if a[mid] >= x:
            r = mid
        else:
            l = mid + 1
    if a[l] == x:
        print("{} {}".format(a[i], a[l]))

# ver. 2
for i in range(n-1):
    # bisearch, a[k] = m - a[i]
    x = m - a[i]
    # search from a[i+1] to a[end-]
    p = bisect_left(a, x, lo=i+1, hi=n)
    if a[p] == x:
        print("{} {}".format(a[i], a[p]))
\end{lstlisting}

\newpage

\subsubsection{整数二分\ 最大值最小化}

\textbf{序列划分: 二分 + 贪心}

给定一个序列, 如\{2, 2, 3, 4, 5, 1\}, 将其划分为m个连续的子序列$S_1, S_2, S_3$, 每个子序列至少有一个元素, 使得每个子序列的和的最大值最小

EX. m = 3

划分为(2, 2, 3), (4, 5), (1) 子序列和分别为7, 9, 1, 最大值为9

划分为(2, 2, 3), (4), (5, 1) 子序列和为7, 4, 6, 最大值为7, 优于前一个

\begin{lstlisting}[style = Python]
# Input:
# array
# m: amount of subarray
# Output:
# x: minimum of maximum of sum(all possible subarrays)
# subarrays

from bisect import *

a = list(map(int, input().split()))
m = int(input())
n = len(a)
l, r = max(a), sum(a)
subs = []

while l < r:
    mid = l + (r - l) // 2
    # greedy divide subarray
    # each sum(subarray) <= mid
    flag = False
    idx = 0
    subs = []
    for i in range(m):
        sum = 0
        while idx < n and sum + a[idx] <= mid:
            sum += a[idx]
            idx +=1
        if idx == n or sum + a[idx] > mid:
            subs.append(idx-1)
    # judge
    if subs[-1] < n-1:
        flag = False
    else:
        flag = True
    # binary control
    if flag:
        # reduce
        r = mid
    else:
        # enlarge
        l = mid + 1

print('minimum sum: ', l)
for i in range(m):
    if i == 0:
        print(a[ : subs[0]+1])
    else:
        print(a[subs[i-1]+1 : subs[i]+1])
\end{lstlisting}

\newpage

\subsubsection{整数二分\ 最小值最大化}

\textbf{洛谷P1824 进击的奶牛}

在一条很长的直线上, 指定n个坐标点. 有c头牛, 每头牛占据一个坐标点, 求相邻两头牛之间距离的最大值

Input:

第1行输入: n c

第2行开始每行输入: 一个整数, 表示每个点的坐标

\begin{lstlisting}[style = Python]
n, c = map(int, input().split())
x = []
for i in range(n):
    x.append(int(input()))
x.sort()

def check(x, dis, c):
    i = 1
    last = 0
    c -= 1
    while c > 0 and i < len(x):
        while i < len(x) and x[i] - x[last] < dis:
            i += 1
        if i < len(x) and x[i] - x[last] >= dis:
            c -= 1
            last = i
            i += 1
    if c == 0:
        return True
    else:
        return False

l, r = 0, x[-1] - x[0]
ans = 0
while l < r:
    mid = l + (r - l) // 2
    if check(x, mid, c):
        ans = mid
        l = mid + 1
    else:
        r = mid
print(ans)
\end{lstlisting}

\subsubsection{实数二分}

\textbf{for控制}: 过大的for次数会超时, 过小的会导致精度不够答案错误. 一般取100, 但是循环体内计算量大的时候容易超时, 可以缩减到50

\textbf{while控制}: while需要设计精度eps, 过小的eps会超时, 过大的会导致精度不够答案错误

\begin{lstlisting}[style = Python]
# 精度, 可以调整
eps = 1e-7
while r - l > eps:
# for ver.
# epoch = 100 # 轮次
# for i in range(epochs):
    mid = l + (r - l) / 2.0
    if check(mid):
        # reduce range
        r = mid
    else:
        # enlarge range
        l = mid
return l
\end{lstlisting}

\newpage

\textbf{分蛋糕 poj 3122}

m+1个人分n个半径不同的蛋糕, 要求每个人分得的蛋糕重量一致, 且必须是可以切出来的一整块, 每个人能分到的最大蛋糕是多少.

Input:

第一行: 1个整数, 表示测试用例个数

对每个测试, 第一行输入n, m. 第二行输入n个整数, 表示每个蛋糕的半径

Output:

对于每个测试, 输出一个答案, 保留4位小数

可以将问题建模为最小值最大化问题, 用面积代替重量

\textbf{保留小数}

\begin{lstlisting}[style = Python]
a = 1.13456 # float
ans = format(a, '.4f) # 四舍五入保留4位小数
\end{lstlisting}

\begin{lstlisting}[style = Python]
def check(mid, area, f):
    sum = 0
    for i in range(len(area)):
        sum += int(area[i] / mid)
    if sum >= f:
        return True
    else:
        return False

eps = 1e-5
pi = 3.1415926
T = int(input())
for t in range(T):
    n, f = map(int, input().split())
    cakes = list(map(float, input().split()))
    maxx = 0
    area = []
    for i in range(n):
        area.append(pi * cakes[i] * cakes[i])
        if area[i] > maxx:
            maxx = area[i]
    l, r = 0, maxx
    while r - l > eps:
        mid = l + (r - l) / 2.0
        if check(mid, area, f):
            l = mid
        else:
            r = mid
    ans = format(l, '.4f')
    print(ans)
\end{lstlisting}

\subsection{三分法}

用于求取单峰函数的极值. 通过在[l, r]内取两个点\textbf{mid1, mid2}, 将函数分为\textbf{三段}

\begin{lstlisting}[style = Python]
k = (r - l) / 3.0
mid1, mid2 = l + k, r - k
\end{lstlisting}

\newpage

\textbf{三分法模板\ 洛谷P3382}

给出一个N次多项式函数, 保证在区间[l, r]内存在一点x, 使得x是函数在区间上的极大值, 求出x

Input:

第一行输入n: N l r

第二行输入: N + 1个实数, 表示从高到低各项的系数

Output:

x, 四舍五入保留5位小数

\begin{lstlisting}[style = Python]
n, l, r = map(float, input().split())
n = int(n)
cons = list(map(float, input().split()))
def f(x, n, cons):
    sum = 0
    for i in range(1, n+2):
        sum += cons[-i] * pow(x, i-1)
    return sum
eps = 1e-6
while r - l > eps:
    k = (r - l) / 3.0
    mid1, mid2 = l + k, r - k
    if f(mid1, n, cons) < f(mid2, n, cons):
        l = mid1
    else:
        r = mid2
ans = format(l, '.5f')
print(ans)
\end{lstlisting}

\newpage

\textbf{三分法 函数 洛谷P1883}

给定 $n$ 个二次函数 $f_1(x),f_2(x),\dots,f_n(x)$(均形如 $ax^2+bx+c$),设 $F(x)=\max\{f_1(x),f_2(x),...,f_n(x)\}$,求 $F(x)$ 在区间 $[0,1000]$ 上的最小值。

输入格式

输入第一行为正整数 $T$,表示有 $T$ 组数据。

每组数据第一行一个正整数 $n$,接着 $n$ 行,每行 $3$ 个整数 $a,b,c$,用来表示每个二次函数的 $3$ 个系数,注意二次函数有可能退化成一次。

输出格式

每组数据输出一行,表示 $F(x)$ 的在区间 $[0,1000]$ 上的最小值。答案精确到小数点后四位,四舍五入。

输入输出样例 1

输入 1

2

1

2 0 0

2

2 0 0

2 -4 2

输出 1


0.0000

0.5000


说明/提示

对于 $50\%$ 的数据,$n\le 100$。

对于 $100\%$ 的数据,$T<10$,$\ n\le 10^4$,$0\le a\le 100$,$|b| \le 5\times 10^3$,$|c| \le 5\times 10^3$。

\begin{lstlisting}[style = Python]
def cal(cons, x):
    ans = -float('inf')
    for a, b, c in cons:
        ans = max(ans, a * x ** 2 + b * x + c)
    return ans

eps = 1e-9
T = int(input())
for t in range(T):
    n = int(input())
    cons = []
    for i in range(n):
        a, b, c = map(float, input().split())
        cons.append([a, b, c])
    l = 0
    r = 1000
    while r - l > eps:
        margin = (r - l) / 3.0
        mid1 = l + margin
        mid2 = r - margin
        f1 = cal(cons, mid1)
        f2 = cal(cons, mid2)
        if f1 <= f2:
            r = mid2
        else:
            l = mid1
    ans = cal(cons, l)
    print(format(ans, '.4f'))
\end{lstlisting}

\newpage

\subsection{排序与排列}

\subsubsection{排序}

\textbf{Python自定排序key}

\begin{verbatim}
key function:
Input: x, y
Output:  1:  x > y
         0:  x = y
         -1: x < y
\end{verbatim}

\begin{lstlisting}[style = Python]
from functools import cmp_to_key
def cmp(x, y):
    # compare x and y
    # return result
sort_key = cmp_to_key(cmp)
\end{lstlisting}

\textbf{洛谷 P1093}

P1093 [NOIP 2007 普及组] 奖学金

题目背景

NOIP2007 普及组 T1

题目描述

某小学最近得到了一笔赞助,打算拿出其中一部分为学习成绩优秀的前 $5$ 名学生发奖学金。期末,每个学生都有 $3$ 门课的成绩:语文、数学、英语。先按总分从高到低排序,如果两个同学总分相同,再按语文成绩从高到低排序,如果两个同学总分和语文成绩都相同,那么规定学号小的同学排在前面,这样,每个学生的排序是唯一确定的。

任务:先根据输入的 $3$ 门课的成绩计算总分,然后按上述规则排序,最后按排名顺序输出前五名名学生的学号和总分。

注意,在前 $5$ 名同学中,每个人的奖学金都不相同,因此,你必须严格按上述规则排序。例如,在某个正确答案中,如果前两行的输出数据(每行输出两个数:学号、总分) 是:

7 279  

5 279


这两行数据的含义是:总分最高的两个同学的学号依次是 $7$ 号、$5$ 号。这两名同学的总分都是 $279$ (总分等于输入的语文、数学、英语三科成绩之和) ,但学号为 $7$ 的学生语文成绩更高一些。

如果你的前两名的输出数据是:

5 279  

7 279


则按输出错误处理,不能得分。

输入格式

共 $n+1$ 行。

第 $1$ 行为一个正整数 $n \le 300$,表示该校参加评选的学生人数。

第 $2$ 到 $n+1$ 行,每行有 $3$ 个用空格隔开的数字,每个数字都在 $0$ 到 $100$ 之间。第 $j$ 行的 $3$ 个数字依次表示学号为 $j-1$ 的学生的语文、数学、英语的成绩。每个学生的学号按照输入顺序编号为 $1\sim n$(恰好是输入数据的行号减 $1$)。

保证所给的数据都是正确的,不必检验。

输出格式

共 $5$ 行,每行是两个用空格隔开的正整数,依次表示前 $5$ 名学生的学号和总分。

\newpage

输入输出样例 1

输入 1

\begin{verbatim}
6
90 67 80
87 66 91
78 89 91
88 99 77
67 89 64
78 89 98
\end{verbatim}

输出 1

\begin{verbatim}
6 265
4 264
3 258
2 244
1 237
\end{verbatim}

\begin{lstlisting}[style = Python]
# 洛谷P1093

# 自定义比较函数 cmp -> key
# x: (id, c, m, e, sum)
# Input: x, y
# Output:
# 1: x > y; 0: x = y; -1: x < y
from functools import cmp_to_key
def cmp(x, y):
    if x[4] > y[4]:
        return 10
    elif x[4] < y[4]:
        return -1
    else:
        if x[1] > y[1]:
            return 1
        elif x[1] < y[1]:
            return -1
        else:
            if x[0] < y[0]:
                return 1
            else:
                return -1
sort_key = cmp_to_key(cmp)

n = int(input())
stu = []
for id in range(1, n+1):
    c, m, e = map(int, input().split())
    summ = c + m + e
    stu.append([id, c, m, e, summ])
rst = sorted(stu, key=sort_key, reverse=True)
for i in range(5):
    print('{} {}'.format(rst[i][0], rst[i][-1]))
\end{lstlisting}

\newpage
\subsubsection{排列}

\textbf{Python 生成全排列}

\begin{lstlisting}[style = Python]
from itertools import permutations
a = [1, 2, 3]
# 全排列
per = permutations(a)
for item in per:
    print(item)
# Output:
(1, 2, 3)
(1, 3, 2)
(2, 1, 3)
(2, 3, 1)
(3, 1, 2)
(3, 2, 1)

# n取m排列
per = permutations(a, 2)
for item in per:
    print(item)
# Output:
(1, 2)
(1, 3)
(2, 1)
(2, 3)
(3, 1)
(3, 2)
\end{lstlisting}

\newpage
\textbf{蓝桥杯2016 省赛 Cpp A组 T6}

有加减乘除4种运算:

$() + () = ()$

$() - () = ()$

$() \times () = ()$

$() \div () = ()$

每个括号代表1-13中的某个数字, 但不能重复, 一共有多少种方案

\begin{lstlisting}[style = Python]
# 蓝桥杯2016省赛C++A组第6题

a = [1, 2, 3, 4, 5, 6, 7, 8, 9, 10, 11, 12, 13]
n = len(a)
vis = [False] * n
b = [None] * n
ans = 0

def permu(s, t):
    global ans
    # 输出
    if s == 12 and b[9] == b[10] * b[11]:
        ans += 1
        return
    # 剪枝
    if s == 3 and b[0] + b[1] != b[2]:
        return
    if s == 6 and b[3] - b[4] != b[5]:
        return
    if s == 9 and b[6] * b[7] != b[8]:
        return
    # 全排列
    for i in range(t):
        if not vis[i]:
            vis[i] = True
            b[s] = a[i]
            permu(s+1, t)
            vis[i] = False

permu(0, n)
print(ans)
\end{lstlisting}

\newpage
\subsection{分治法}

将问题分解为多个独立的子问题, 子问题的规模大致相等. 分治法常能够显著降低算法的时间复杂度.

分治法经典应用: 汉诺塔, 归并排序, 快速排序 等

\textbf{汉诺塔\ 蓝桥1512}

Input: 两个整数 N M. N为要移动的盘子数, M为最少移动步骤的第M步

Output: 

第一行: \#No: a->b. No为移动的盘子编号, a->b为从a杆移动到b杆, 取值为\{A, B, C\}

第二行: 一个整数, 最少移动步数

\begin{lstlisting}[style = Python]
step = 0
n, m = map(int, input().split())

def hanoi(x, y, z, n):
    global step, m
    if n == 1:
        step += 1
        if step == m:
            print('#{}: {}->{}'.format(m, x, z))
    else:
        hanoi(x, z, y, n-1)
        step += 1
        if step == m:
            print('#{}: {}->{}'.format(m, x, z))
        hanoi(y, x, z, n-1)

hanoi('A', 'B', 'C', n)
print(step)
\end{lstlisting}

\newpage
\textbf{逆序对 hdu 4911}

输入一个序列, 交换任意两个元素, 不超过k次交换之后, 最少的逆序对有多少个.

逆序对: $a_i, a_j,\ when\ 1 <= i < j < n\ and\ a_i > a_j$ 

Input: 

第一行: n, k

第二行: n个整数

Output: 最少的逆序对数量

\begin{lstlisting}[style = Python]
# hdu 4911

n, k = map(int, input().split())
a = list(map(int, input().split()))
count = 0
b = [0]*10000

def merge(l, mid, r):
    global count, a, b
    i, j = l, mid+1
    t = 0
    while i <= mid and j <= r:
        if a[i] > a[j]:
            b[t] = a[j]
            t, j = t+1, j+1
            count += mid - i + 1
        else:
            b[t] = a[i]
            t, i = t+1, i+1
    while i <= mid:
        b[t] = a[i]
        t, i = t+1, i+1
    while j <= r:
        b[t] = a[j]
        t, j = t+1, j+1
    for i in range(t):
        a[l+i] = b[i]

def merge_sort(l, r):
    if l < r:
        mid = (l + r) // 2
        merge_sort(l, mid)
        merge_sort(mid+1, r)
        merge(l, mid, r)

merge_sort(0, len(a)-1)
if count <= k:
    print(0)
else:
    print(count - k)
\end{lstlisting}

\newpage
\subsection{ST算法 倍增法}

\textbf{倍增法原理}

每一步都以2倍扩展区间, 以此快速覆盖整个区间. 在编程上, 不使用\verb*|*2|, 而是利用二进制数的特性, 将一个数N用二进制展开 $N = a_0 2^0 + a_1 2^1 + \dots$, 这样, 整数n的数位只有$log_2 n$位, 以每一位作为一个跳完下一个状态的跳板, 跳板数量等同于二进制位置, 只有$log_2 n$

\textbf{ST算法}

ST算法是用于求解RMQ(Range Minimum/Maximum Query, 区间最值查询)的算法, 基于倍增法原理, 适用于\textbf{静态空间}.

ST算法的基本思想: 对于一个区间[a, b], 区间上的最值由两个子区间[c, d], [e, f]决定, 且$[c, d] \vee [e, f] = [a, b]$. 即$min\{[a, b]\} = min\{min\{[c, d]\}, min\{[e, f]\}\}$

\textbf{Procedure}

1. 将数列按照倍增法划分为多个小区间. 

第一组区间长度为1; 第二组区间长度为2; 第三组区间长度为4, 以此类推. 每一组区间中, 第i+1个区间的左端点是第i个区间的左端点+1.

以此可以通过不同组的区间, 递推出每一组子区间的最值. 例如, 第三组第一个子区间\{1, 2, 3, 4\}可以通过第二组的\{1, 2\}\{3, 4\}递推得到, 且有递推公式.

\begin{verbatim}
min: dp[s][k] = min{dp[s][k-1], dp[s + 1 << (k-1)][k-1]}
max: dp[s][k] = max{dp[s][k-1], dp[s + 1 << (k-1)][k-1]}
\end{verbatim}

其中, dp[s][k]表示左端点为s, 区间长度为$2^k$的区间最值, \verb|1 << (k-1|表示$2^{k-1}$

2. 查询任意区间的最值

对于区间[L, R], 为保证覆盖, 需要两个长度为x的子区间, 且满足$x \leq len,\ 2x \geq len$. 

计算dp, 根据dp[s][k]定义, $2^k = x$, 因此$k = \lfloor log_2(len) \rfloor$

区间[L, R]的最小值, 为覆盖其的小区间的最小值, 即

\verb|min = min(dp_min[L][k], dp_min[R - (1 << k) + 1][k])|

\verb|max = max(dp_max[L][k], dp_max[R - (1 << k) + 1][k])|


\textbf{模版 洛谷P2880}

给定一个包含n个整数的数列和q个区间查询, 查询区间内最大值和最小值的差.

Input: 第一行输入n, q. 接下来的n行, 每行输入一个整数$h_i$; 再加下来q行, 每行输入两个整数a b, 表示一个区间查询.

\begin{lstlisting}[style = Python]
from math import log2
n, q = map(int, input().split())
# maximum of k
p = int(log2(n))
# RMQ table, 1-index
dp_min = [[0 for _ in range(p+1)]for _ in range(n + 1)]
dp_max = [[0 for _ in range(p+1)]for _ in range(n + 1)]
# read in list, 1-index
a = [-1]
for i in range(n):
    a.append(int(input()))

# initial dp table
# len of sub-range is 1
for i in range(1, n+1):
    dp_min[i][0] = a[i]
    dp_max[i][0] = a[i]
# len of sub-range > 1
for k in range(1, p+1):
    for s in range(1, n):
        if s + (1 << k) > n + 1:
            break
        dp_min[s][k] = min(dp_min[s][k-1], dp_min[s + (1 << (k-1))][k-1])
        dp_max[s][k] = max(dp_max[s][k-1], dp_max[s + (1 << (k-1))][k-1])

# RMQ
for _ in range(q):
    L, R = map(int, input().split())
    k = int(log2(R - L + 1))
    max_query = max(dp_max[L][k], dp_max[R - (1 << k) + 1][k])
    min_query = min(dp_min[L][k], dp_min[R - (1 << k) + 1][k])
    print(max_query - min_query)    
\end{lstlisting}

\newpage
\subsection{离散化}

离散化适用场景: 给出一个数列, 问题要求关注数字的相对大小, 数字的绝对大小不重要. 此时使用离散化, 在列表中, 使用数字的相对大小排序取代数字的数值.

\textbf{Procedure}

1. 排序

2. 离散化: 将排序后的数列元素从1开始分配数值

3. 归位: 将数列元素重新分配回原始位置

\begin{lstlisting}[style = Python]
from random import randint

a = [] # element[0]: val, element[1]: original place
for i in range(10):
    a.append([randint(1, 5000), i])

print('original list:')
print(a)

# main function
# sort
a.sort(key=lambda x : x[0])
n = len(a)
# generate new list
newa = [0 for _ in range(n)]
for i in range(n):
    newa[a[i][1]] = i + 1

print('processed list:')
print(newa)

# IF: same value should be corresponded to the same processed value
print('---------------same value -> same order value---------------')
a = [[13, 0],[65, 1], [8, 2], [13, 3], [93, 4], [197, 5]]
print('original list')
print(a)
a.sort(key=lambda x : x[0])
n = len(a)
# generate new list
newa = [0 for _ in range(n)]
for i in range(n):
    if i > 0 and a[i][0] == a[i-1][0]:
        newa[a[i][1]] = newa[a[i-1][1]]
        continue
    newa[a[i][1]] = i + 1
print('processed list:')
print(newa)
\end{lstlisting}

\begin{verbatim}
original list:
[[2700, 0], [1266, 1], [1999, 2], [3798, 3], [1139, 4], [3214, 5], 
[4171, 6], [4597, 7], [3040, 8], [1135, 9]]
processed list:
[5, 3, 4, 8, 2, 7, 9, 10, 6, 1]
---------------same value -> same order value---------------
original list
[[13, 0], [65, 1], [8, 2], [13, 3], [93, 4], [197, 5]]
processed list:
[2, 4, 1, 2, 5, 6]
\end{verbatim}




\end{sloppy}
\end{document}