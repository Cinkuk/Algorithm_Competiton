\documentclass{ctexart}

\usepackage{graphicx} % Required for inserting images
\usepackage{listings}
\usepackage{geometry}
\usepackage{graphicx}
\usepackage{xcolor}
\usepackage{amsmath}
\usepackage{bm}
\usepackage{bbding}
\usepackage{markdown}
\markdownSetup{hybrid = true}

\geometry{a4paper, left=1.45cm, right=1.45cm, top=1.45cm, bottom=1.45cm}
\geometry{hcentering}

\lstdefinelanguage{Pseudocode}{
    keywords = {IF, ELSEIF, ELSE, THEN, ENDIF, FOR, ENDFOR, WHILE, ENDWHILE, FROM, TO, DO, END, Function, return, true, false, NOT, INPUT, OUTPUT, None},
    comment = [l]{//},
    morecomment = [s]{/*}{*/}
}

\lstdefinestyle{Pseudocode}{
    language        =   Pseudocode,
    basicstyle      =   \ttfamily\footnotesize,
    keywordstyle    =   \rmfamily\tiny\bfseries\underbar,
    stringstyle     =   \itshape,
    commentstyle    =   \itshape,
    numbers         =   left,
    numberstyle     =   \tiny,
    numbersep       =   1em,
    breaklines      =   true,
    columns         =   flexible,
    frame           =   tb,
    framesep        =   0.5em,
    basewidth       =   0.3em,
    tabsize         =   4,
}

\lstdefinestyle{Python}{
    language        =   Python,
    basicstyle      =   \ttfamily\footnotesize,
    keywordstyle    =   \rmfamily\bfseries,
    stringstyle     =   \itshape,
    commentstyle    =   \itshape,
    numbers         =   left,
    numberstyle     =   \tiny,
    numbersep       =   1em,
    breaklines      =   true,
    columns         =   flexible,
    frame           =   tb,
    framesep        =   0.5em,
    basewidth       =   0.3em,
    tabsize         =   4,
}

\lstdefinestyle{Cpp}{
    language        =   C++,
    basicstyle      =   \ttfamily\footnotesize,
    keywordstyle    =   \rmfamily\bfseries,
    stringstyle     =   \itshape,
    commentstyle    =   \itshape,
    numbers         =   left,
    numberstyle     =   \tiny,
    numbersep       =   1em,
    breaklines      =   true,
    columns         =   flexible,
    frame           =   tb,
    framesep        =   0.5em,
    basewidth       =   0.3em,
    tabsize         =   4,
}

\lstdefinestyle{MATLAB}{
    language        =   MATLAB,
    basicstyle      =   \ttfamily\footnotesize,
    keywordstyle    =   \rmfamily\bfseries,
    stringstyle     =   \itshape,
    commentstyle    =   \itshape,
    numbers         =   left,
    numberstyle     =   \tiny,
    numbersep       =   1em,
    breaklines      =   true,
    columns         =   flexible,
    frame           =   tb,
    framesep        =   0.5em,
    basewidth       =   0.3em,
    tabsize         =   4,
}


\title{算法竞赛笔记}
\author{Chunyin Chan}

\begin{document}
\begin{sloppy}
    \maketitle
    \setcounter{page}{0}
    \thispagestyle{empty}

    \newpage
    \setcounter{page}{0}
    \tableofcontents
    \thispagestyle{empty}

    \section{基础数据结构}

    \subsection{Link List}
        
    \subsubsection{link list in python:}

    \begin{lstlisting}[style = Python]
    class nodes:
        def __init__(self, val=None, pre=None, next=None):
            self.val = val
            self.next = next
    \end{lstlisting}

    \textbf{EX 1}

    \begin{markdown}
    洛谷 P1996 约瑟夫问题

    题目描述

    $n$ 个人围成一圈,从第一个人开始报数,数到 $m$ 的人出列,再由下一个人重新从 $1$ 开始报数,数到 $m$ 的人再出圈,依次类推,直到所有的人都出圈,请输出依次出圈人的编号。

    输入格式

    输入两个整数 $n,m$。

    输出格式

    输出一行 $n$ 个整数,按顺序输出每个出圈人的编号。

    输入输出样例 #1

    输入 #1

    ```
    10 3
    ```

    输出 #1

    ```
    3 6 9 2 7 1 8 5 10 4
    ```

    说明/提示

    $1 \le m, n \le 100$
    \end{markdown}



    \begin{lstlisting}[style = Python]
        # 动态链表
        class node:
            data = None
            next = None
        
        n, m = map(int, input().split())
        # init link list
        head = node()
        head.data = 1
        now = head
        for i in range(2, n+1):
            p = node()
            p.data = i
            now.next = p
            now = p
        now.next = head
        prev = now
        now = head
        while n > 0:
            n -= 1
            for i in range(m-1):
                prev = now
                now = now.next
            print(now.data, end=' ')
            prev.next = now.next
            now = now.next
    \end{lstlisting}

    \begin{lstlisting}[style = Python]
        # 列表和索引计算
        n, m = map(int, input().split())
        people = list(range(1, n+1))
        rst = []
        current = 0
        index = 0
        while people:
            index = (current + m - 1) % len(people)
            rst.append(people.pop(index))
            current = index
        print(' '.join(map(str, rst)))
    \end{lstlisting}

    \textbf{EX 2}

    \begin{markdown}
    洛谷 P1160 队列安排

    题目描述

    一个学校里老师要将班上 $N$ 个同学排成一列,同学被编号为 $1\sim N$,他采取如下的方法:

    1. 先将 $1$ 号同学安排进队列,这时队列中只有他一个人;

    2. $2\sim N$ 号同学依次入列,编号为 $i$ 的同学入列方式为:老师指定编号为 $i$ 的同学站在编号为 $1\sim(i-1)$ 中某位同学(即之前已经入列的同学)的左边或右边;

    3. 从队列中去掉 $M$ 个同学,其他同学位置顺序不变。

    在所有同学按照上述方法队列排列完毕后,老师想知道从左到右所有同学的编号。

    输入格式

    第一行一个整数 $N$,表示了有 $N$ 个同学。

    第 $2\sim N$ 行,第 $i$ 行包含两个整数 $k,p$,其中 $k$ 为小于 $i$ 的正整数,$p$ 为 $0$ 或者 $1$。若 $p$ 为 $0$,则表示将 $i$ 号同学插入到 $k$ 号同学的左边,$p$ 为 $1$ 则表示插入到右边。

    第 $N+1$ 行为一个整数 $M$,表示去掉的同学数目。

    接下来 $M$ 行,每行一个正整数 $x$,表示将 $x$ 号同学从队列中移去,如果 $x$ 号同学已经不在队列中则忽略这一条指令。

    输出格式

    一行,包含最多 $N$ 个空格隔开的整数,表示了队列从左到右所有同学的编号。

    输入输出样例 #1

    输入 #1

    ```
    4
    1 0
    2 1
    1 0
    2
    3
    3
    ```

    输出 #1

    ```
    2 4 1
    ```

    说明/提示

    **【样例解释】**

    将同学 $2$ 插入至同学 $1$ 左边,此时队列为:

    `2 1`

    将同学 $3$ 插入至同学 $2$ 右边,此时队列为:

    `2 3 1`  

    将同学 $4$ 插入至同学 $1$ 左边,此时队列为:

    `2 3 4 1`  

    将同学 $3$ 从队列中移出,此时队列为:

    `2 4 1`  

    同学 $3$ 已经不在队列中,忽略最后一条指令

    最终队列:

    `2 4 1`  

    **【数据范围】**

    对于 $20\%$ 的数据,$1\leq N\leq 10$。

    对于 $40\%$ 的数据,$1\leq N\leq 1000$。

    对于 $100\%$ 的数据,$1<M\leq N\leq 10^5$。
    \end{markdown}

    \begin{lstlisting}[style = Python]
        # 链表
        class nodes:
            def __init__(self, val=None, next=None, prev=None):
                self.val = val
                self.next = next
                self.prev = prev
        
        N = int(input())
        people = nodes(1)
        head = people
        idx = {1: people}
        for i in range(2, N+1):
            node = nodes(i)
            k, p = map(int, input().split())
            k = idx[k]
            if p == 0:
                if k.prev:
                    k.prev.next = node
                    node.prev = k.prev
                node.next = k
                k.prev = node
                if k == head:
                    head = node
            elif p == 1:
                if k.next:
                    k.next.prev = node
                    node.next = k.next
                node.prev = k
                k.next = node
            idx[i] = node
            
        M = int(input())
        for i in range(M):
            x = int(input())
            if x in idx.keys():
                k = idx[x]
                if k.prev:
                    k.prev.next = k.next
                    if k.next:
                        k.next.prev = k.prev
                if k.next:
                    k.next.prev = k.prev
                    if k.prev:
                        k.prev.next = k.next
                    if k == head:
                        head = k.next
                del idx[x]
        
        while head:
            print(head.val, end=' ')
            head = head.next
    \end{lstlisting}

    \subsection{Queue}

    \subsubsection{Queue in Python}

    \textbf{双端队列}

    \begin{lstlisting}[style = Python]
    from collections import deque # 双端队列

    queue = deque(maxlen = 10) # 最大长度为10, None为无限制
    queue = deque(iterable) # init queue by iterable obj.

    queue.append(x) # add x to right side
    queue.appendleft(x) # add x to left side

    queue.extend(iterable) # extend right side by iterable obj.
    queue.extendleft(iterable) # extend the left side by iterable obj.

    queue.pop() # pop element from right side 
    queue.popleft() # pop element from left side

    queue.remove(x) # remove x from Queue, raise error if not found
    queue.clear() # clear Queue

    queue.reverse() # reverse Queue
    queue.insert(i, x) # insert x into queue at position i
    queue.count(x) # count the number of queue elements = x
    queue.index(x) # return first match position, raise error if not found

    \end{lstlisting}

    \textbf{EX 1}

    \begin{markdown}
    洛谷 P1540 [NOIP 2010 提高组] 机器翻译

    题目背景

    NOIP2010 提高组 T1

    题目描述

    小晨的电脑上安装了一个机器翻译软件,他经常用这个软件来翻译英语文章。

    这个翻译软件的原理很简单,它只是从头到尾,依次将每个英文单词用对应的中文含义来替换。对于每个英文单词,软件会先在内存中查找这个单词的中文含义,如果内存中有,软件就会用它进行翻译;如果内存中没有,软件就会在外存中的词典内查找,查出单词的中文含义然后翻译,并将这个单词和译义放入内存,以备后续的查找和翻译。

    假设内存中有 $M$ 个单元,每单元能存放一个单词和译义。每当软件将一个新单词存入内存前,如果当前内存中已存入的单词数不超过 $M-1$,软件会将新单词存入一个未使用的内存单元;若内存中已存入 $M$ 个单词,软件会清空最早进入内存的那个单词,腾出单元来,存放新单词。

    假设一篇英语文章的长度为 $N$ 个单词。给定这篇待译文章,翻译软件需要去外存查找多少次词典?假设在翻译开始前,内存中没有任何单词。

    输入格式

    共 $2$ 行。每行中两个数之间用一个空格隔开。

    第一行为两个正整数 $M,N$,代表内存容量和文章的长度。

    第二行为 $N$ 个非负整数,按照文章的顺序,每个数(大小不超过 $1000$)代表一个英文单词。文章中两个单词是同一个单词,当且仅当它们对应的非负整数相同。

    输出格式

    一个整数,为软件需要查词典的次数。

    输入输出样例 #1

    输入 #1

    ```
    3 7
    1 2 1 5 4 4 1
    ```

    输出 #1

    ```
    5
    ```

    说明/提示

    样例解释

    整个查字典过程如下:每行表示一个单词的翻译,冒号前为本次翻译后的内存状况:

    1. `1`:查找单词 1 并调入内存。
    2. `1 2`:查找单词 2 并调入内存。
    3. `1 2`:在内存中找到单词 1。
    4. `1 2 5`:查找单词 5 并调入内存。
    5. `2 5 4`:查找单词 4 并调入内存替代单词 1。
    6. `2 5 4`:在内存中找到单词 4。
    7. `5 4 1`:查找单词 1 并调入内存替代单词 2。

    共计查了 $5$ 次词典。

    数据范围

    - 对于 $10\%$ 的数据有 $M=1$,$N \leq 5$;
    - 对于 $100\%$ 的数据有 $1 \leq M \leq 100$,$1 \leq N \leq 1000$。
    \end{markdown}

    \begin{lstlisting}[style = Python]
    # 队列 collections.deque
    from collections import deque
    from array import array

    M, N = map(int, input().split())
    q = deque(maxlen=M) # 双向队列
    qs = set() # 用作哈希表
    count = 0 # 计数
    inp_ary = array('i', map(int, input().split())) # 输入数据
    l = len(inp_ary)
    for i in range(l):
        if inp_ary[i] not in qs:
            if len(q) == M:
                qs.remove(q.popleft())
            qs.add(inp_ary[i])
            q.append(inp_ary[i])
            count += 1
    print(count)
    \end{lstlisting}

    \textbf{单调队列}

    \begin{markdown}
    # 洛谷 P1886 滑动窗口 /【模板】单调队列

    ## 题目描述

    有一个长为 $n$ 的序列 $a$,以及一个大小为 $k$ 的窗口。现在这个从左边开始向右滑动,每次滑动一个单位,求出每次滑动后窗口中的最大值和最小值。

    例如,对于序列 $[1,3,-1,-3,5,3,6,7]$ 以及 $k = 3$,有如下过程:

    $$\def\arraystretch{1.2}
    \begin{array}{|c|c|c|}\hline
    \textsf{窗口位置} & \textsf{最小值} & \textsf{最大值} \\ \hline
    \verb![1   3  -1] -3   5   3   6   7 ! & -1 & 3 \\ \hline
    \verb! 1  [3  -1  -3]  5   3   6   7 ! & -3 & 3 \\ \hline
    \verb! 1   3 [-1  -3   5]  3   6   7 ! & -3 & 5 \\ \hline
    \verb! 1   3  -1 [-3   5   3]  6   7 ! & -3 & 5 \\ \hline
    \verb! 1   3  -1  -3  [5   3   6]  7 ! & 3 & 6 \\ \hline
    \verb! 1   3  -1  -3   5  [3   6   7]! & 3 & 7 \\ \hline
    \end{array}
    $$

    ## 输入格式

    输入一共有两行,第一行有两个正整数 $n,k$。
    第二行 $n$ 个整数,表示序列 $a$

    ## 输出格式

    输出共两行,第一行为每次窗口滑动的最小值   
    第二行为每次窗口滑动的最大值

    ## 输入输出样例 #1

    ### 输入 #1

    ```
    8 3
    1 3 -1 -3 5 3 6 7
    ```

    ### 输出 #1

    ```
    -1 -3 -3 -3 3 3
    3 3 5 5 6 7
    ```

    ## 说明/提示

    【数据范围】    
    对于 $50\%$ 的数据,$1 \le n \le 10^5$;  
    对于 $100\%$ 的数据,$1\le k \le n \le 10^6$,$a_i \in [-2^{31},2^{31})$。
    \end{markdown}

    \begin{lstlisting}[style = Python]
    from collections import deque
    from array import array
    import sys

    # input 
    n, k = map(int, input().split())
    a = list(map(int, input().split()))

    min_q = deque()
    max_q = deque()
    min_rst = []
    max_rst = []

    for i in range(n):
        # minimum queue
        while min_q and a[min_q[-1]] > a[i]:
            min_q.pop()
        min_q.append(i)
        # maximum queue
        while max_q and a[max_q[-1]] < a[i]:
            max_q.pop()
        max_q.append(i)

        # 开始记录
        if i >= k - 1:
            min_rst.append(a[min_q[0]])
            max_rst.append(a[max_q[0]])
        
        # 弹出窗口外的元素
        if min_q and min_q[0] <= i - k + 1:
            min_q.popleft()
        if max_q and max_q[0] <= i - k + 1:
            max_q.popleft()

    # output
    print(' '.join(map(str, min_rst)))
    print(' '.join(map(str, max_rst)))
    \end{lstlisting}

    
    



    


\end{sloppy}
\end{document}